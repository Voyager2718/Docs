%!Mode::"TeX:UTF-8"
\documentclass[utf-8,12pt]{ctexart}%还可以用ctexbook ctexrep等
\usepackage{amsmath}
\usepackage{xltxtra}
\usepackage{mflogo,texnames}
\usepackage[colorlinks,
            linkcolor=black,
            urlcolor=black
            ]{hyperref}
\usepackage{fancyhdr}
\title{\textbf{PCL公开内容许可 V1}\\\normalsize{Public Contents License V1}}
\begin{document}
\maketitle
\tableofcontents
\pagestyle{fancy}
\newpage{}
\paragraph{}
\noindent 本许可:

允许所有人复制和发布本许可文件的完整版本,但不允许对它进行任何修改

不承担由于本许可所造成的一切责任
\section{导言}
PCL许可是为了保护公开内容作者权利而制定的针对公开内容的一份授权许可协议。

通过此授权许可协议,公开内容作者可以:(1)声明公开内容的版权;(2)声明公开内容使用权的范围。

如未额外声明,默认使用在公开内容首次发布时最新版的PCL许可。
\section{定义}
\begin{itemize}
\item “本许可”:PCL通用内容许可。
\item “公开内容”:包括但不限于所有对非商业用户免费的软件、源代码、文章、书籍、图画等。
\item “修改”:包括但不限于增删或修改作者标识或名字、对公开内容进行增删或修改程序源代码、增删或修改文章、书籍或漫画内容等等。
\item “商业用户”:注册为营利组织的机构或个人,或一年内曾经通过经营获得收入的机构或个人,或有计划通过任何手段获得收入的机构或个人。
\item “非商业用户”:不注册为商业组织且从使用公开内容起的六个月内没有营利计划的机构或个人。
\item “传播”:使用公开内容做任何如果没有许可就会在适用的版权法下直接或间接侵权的事情。传播也包括向公众共享的行为。
\item “向公众共享”:包括但不限于:发布到个人网站、博客、限制浏览的网络平台等,发布到书刊、报纸等,在公众场合演示、放映等。
\item “发布”:公开到任何大众可以浏览的媒体上。
\end{itemize}
\section{非商业用户的许可}
\subsection{使用及传播}
本许可授权非商业用户以阅读、浏览、在软件或网页中使用等形式使用公开内容,同时也授权非商业用户进行传播。

在传播时,非商业用户必须在清晰可见的地方列出公开内容的作者,保持完整的免责声明以及本许可,且保留作者声明需要保留的内容。
\subsection{修改}
\subsubsection{计算机程序或源代码}
本许可授权非商业用户增删或修改公开内容中的计算机程序或源代码。但不能移除作者信息、免责声明、本许可以及作者声明需要保留的信息。

发布修改过的程序或源代码时,必须包含明确的通告说明用户进行了修改。
\subsubsection{文章、书籍、图画等}
本许可禁止非商业用户增删或修改公开内容中的文章、书籍及图画等。
\subsection{用户性质转变}
若非商业用户转变为商业用户,非商业用户的所有许可将同时被收回。用户必须在7天内联系公开内容作者以获得授权许可。

若未能获得授权,用户必须在转变为商业用户的7天内停止修改、传播、发布并删除所有公开内容。
\subsection{许可}
所有基于本许可授予的的权利,只要在所述条件都满足,许可不能被收回。
\section{商业用户的使用许可}
为了使用或传播公开内容,商业用户必须先联系公开内容作者,并在作者授权后使用公开内容。
\section{终止授权}
任何用户在违反本许可中一条或多条规定的同时,授权将被终止。

授权被终止后,用户必须在7天内停止一切使用、传播、发布公开内容的行为,并删除所有的副本。
\section{专利权}
公开内容作者拥有对其作品法律允许范围内的全部专利权。
\section{免责声明}
在适用法律许可下,本许可不对公开内容承担任何担保责任。也不对任何用户使用公开内容时造成的任何后果负责。
\end{document} 