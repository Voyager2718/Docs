%!Mode::"TeX:UTF-8"
\documentclass[UTF8]{ctexart}
\usepackage{amsmath}
\usepackage{xltxtra}
\usepackage{mflogo,texnames}
\usepackage[colorlinks,
            linkcolor=black,
            urlcolor=black
            ]{hyperref}
\usepackage{fancyhdr}
\usepackage{color}
\title{简历}
\author{杨志鹏}
\begin{document}
\maketitle
\pagestyle{fancy}
\tableofcontents{}
\newpage
\section{实用信息}
\begin{itemize}
\item 姓名:杨志鹏\ \ \ 性别:男
\item 出生年月:1992年11月
\item 专业:计算机
\item 学历:本科
\item 留学经历:法国
\item 毕业学校:里尔一大(世界排名约400左右)
\item 电子邮箱:Voyager2718.GitHub@QQ.com或Voyager2718@Gmail.com
\item GitHub:\url{http://voyager2718.github.io}
\item 个人网站:\url{http://changletech.com}
\end{itemize}
\section{个人技能}
\begin{itemize}
\item 网页开发(CSS3/JS/PHP/MySQL/Ajax)\\
例子:\\
个人买的域名:\url{http://changletech.com}\\
个人制作诗赋收集网站:\url{http://changletech.com/poetry}\\
在学校学习网页开发时候做过的例子:\url{http://voyager2718.github.io/Rockets}
\item C\#{}开发Windows桌面应用程序\\
例子:\\
为了在重要的网站上根据网站不同而使用不同密码,本人使用C\#{}设计了一个用MD5加密密码的工具:\url{http://t.cn/R7YfzkB}
\item C/C++开发
\item OCaml
\item Ruby(主要用来使用RGSS)
\item 3D动画:Cinema 4D
\item 视频制作:Premiere/After Effect
\item 文档制作:\LaTeX
\item 粤语/普通话/英语/法语
\item 网页动画制作:Edge Animation
\end{itemize}
※PHP文件在GitHub被设为Private\ Repository,因为PHP文件里有数据库用户和密码等信息。
\section{学习经历}
在法国的本科,我们学习了了C语言、OCaml、VHDL、CSS、PHP、Java、Javascript等。

在学习期间,我们曾经使用ISE在Linux平台上设计过处理器。

曾经使用C\#{}开发不同的Win32程序。

曾经使用MySQL和PHP开发论坛。

同时,我们也学习过网页制作(Javascript/CSS/PHP/MySQL),上面一个介绍火箭的网站即是当时做的。

本人对制作网页和使用C\#{}开发Win32程序比较感兴趣,因为本人比较喜欢制作能和人互动的程序。

在本人驻GoDaddy服务器的网站上,本人曾使用PHPMailer解决无法发送邮件的问题。且有解决使用GoDaddy服务器“被墙”国内无法访问的经验。

曾经使用Ajax做一个权限管理系统。

学余时间,喜欢上知乎、果壳讨论问题。也喜欢上Coursera寻找名校的课程。

在求学路上,因为法国有较多英美留学生的关系,本人曾多次(10次以上)与一个学习经济学的英国人长时间讨论问题。英语口语表达比较流利,能准确说出本人想表达的内容。
\section{工作经历}
在法求学期间,曾经为“法国快运”公司修改并完善其后台管理系统(使用PHP和GoDaddy服务器)。

曾经为国内朋友一个网络展示商品网址开发过一个增删商品的系统(使用PHP)。
\section{未来计划}
希望在不久的将来,能
\begin{itemize}
\item 深入学习微软.NET架构。
\item 深入学习数据挖掘和机器学习。
\item 学习Objective\ C、iOS和Android的开发
\end{itemize}
\section{个人评价}
对层出不穷的新技术充满了好奇和跃跃欲试的心态。虽然学校不要求学习C++、C\#{}、\LaTeX{}、Python、Ruby、Ajax、UE3引擎等技术,但由于本人强烈的好奇心和求知欲,本人在空余时间也多次尝试使用不同的语言解决不同的问题。

每当出现一个问题,即使熬夜到1、2点也要修复并Commit。如果太晚实在无法解决,在第二天也一定会用尽方法解决出现的问题。

如果发现有不完美的地方,(即使是标点符号也)一定会将其修改并Commit。
\section{职位期待}
%由于女朋友在国内,希望在暑假(2015-6中旬至2015-8中旬)的时候回国实习以陪伴女朋友。

对于贵公司,其实本人大约初三(2008年左右)就从广州本土一本叫《数码时代DE》的杂志上获悉。那时候我也经常上糗百,有时候还会上秘密去吐槽一下。也从淘宝买过激活码。6年余伙伴,今天,让我加入一起开发吧!

本人希望在实习的过程中,担任技术开发的工作。

一般来说,本人觉得实习的过程中,薪水都是次要的。本人实习的主要目的是希望能以最快速度适应工作并尽可能多地学习到工作中会用到的经验和知识。因为本人认为,在开发的岗位上,实打实的经验胜过书本上的知识千万倍。在本人日常学习中,我发现书面的知识说是那样,但是可能实战的时候又是完全不同的。所以还是希望能获得尽可能多的工作经验和技能。
\section{兴趣爱好}
本人爱好繁多,主要有以下几个方面:
\begin{itemize}
\item 游泳、骑自行车
\item 看书(主要是政治经济上的书,如《韩非子》《国富论》等)
\item 刻章
\item 练字(硬笔现在主要临摹田英章老师的,软笔主要临摹王羲之大师的)
\item 浏览知乎和果壳(因为这两个网站上都有很多很有趣的各种领域的知识)
\item 化学和物理实验(曾经用硝酸钾和葡萄糖制作KNDX固体火箭发动机)
\item 模型(静态和航模均喜欢)
\end{itemize}
\end{document} 