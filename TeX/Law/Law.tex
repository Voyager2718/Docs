%!Mode::"TeX:UTF-8"
\documentclass[UTF8]{ctexart}
\usepackage{xltxtra}
\usepackage{mflogo,texnames}
\usepackage[colorlinks,
            linkcolor=black,
            urlcolor=black
            ]{hyperref}
\usepackage{fancyhdr}
\title{常用法律及其案例\\\normalsize{第一版}}
\author{杨志鹏}
\begin{document}
\maketitle
\thispagestyle{empty}
\newpage{}
\tableofcontents
\thispagestyle{empty}
\newpage{}
\pagestyle{fancy}
\paragraph{授权许可}
\section{免责声明}
\section{《刑法》}
\subsection{抢劫罪}
\subsection{强奸罪}
\subsection{杀人罪}
\subsection{正当防卫}
\subsubsection{法律条文}
《刑法》第二十条

为了使国家、公共利益、本人或者他人的人身、财产和其他权利免受正在进行的不法侵害,而采取的制止不法侵害的行为,对不法侵害人造成损害的,属于正当防卫,不负刑事责任。

正当防卫明显超过必要限度造成重大损害的,应当负刑事责任,但是应当减轻或者免除处罚。

对正在进行行凶、杀人、抢劫、强奸、绑架以及其他严重危及人身安全的暴力犯罪,采取防卫行为,造成不法侵害人伤亡的,不属于防卫过当,不负刑事责任。

《民法通则》第一百二十八条
因正当防卫造成损害的,不承担民事责任。正当防卫超过必要的限度,造成不应有的损害的,应当承担适当的民事责任。
\subsubsection{解析}
正当防卫有五大必须同时具备的要素,分别是:
\begin{itemize}
\item 有不法侵害发生:发生《刑法》第二十条第三款中规定的不法侵害。
\item 不法侵害正在进行:正当防卫只能在不法侵害正在进行时实行。
\item 防卫行为必须针对加害者本人
\item 必须为了使国家、公共利益、本人或者他人的人身和其他权利免受正在进行的不法侵害:

(1)防卫挑拨,是指为了加害对方,故意以挑衅、引诱等发挑逗他人向自己进攻,然后借口正当防卫加害对方的行为。(2)相互进行的非法侵害行为,是指双方都出于不法侵害的故意而进行的相互侵害行为。需要指出的是,在相互斗殴中,也可能出现正当防卫的前提条件,因而也可能进行正当防卫:一是斗殴一方已经放弃侵害,例如宣布不再斗殴或者认输、求饶、逃跑,而另一方继续侵害;二是在一般性斗殴中,一方突然使用杀伤力很强的凶器,另一方面临生命的严重威胁。
\item 防卫不能明显超过必要限度:

只要造成的损害是制止不法侵害所必需的,即使防卫在强度、后果等方面超过对方可能造成的损害,也不能认为是超过了必要限度。

认为正当防卫的必要限度,就是防卫行为与不法侵害行为在性质、手段、强度和后果上要基本相适应。

必要限度原则上应以制止不法侵害所必需为标准,同时要求防卫行为与不法侵害行为在手段、强度等方面,不存在过于悬殊的差异。
\end{itemize}
\subsubsection{常规案例}
\begin{itemize}
\item A欲强奸B,B捅死A。

B面对A正在进行的强奸,采取防卫行为造成A死亡,不属于防卫过当,不负刑事和民事责任。
\item A欲强奸B,B捅了下A,A倒地,B再次捅A,A死。

防卫过当或故意杀人。A被捅后已经倒地失去继续进行不法侵害的能力。此时B再次捅A则明显不符合「不法侵害正在进行」的必要条件。
\item A欲强奸B,B捅A,未死,但流血,B未报警走开,A死。

如果B捅了A后惊魂未定吓跑了,其不负有救助义务,因此不成立防卫过当,仍属于正当防卫;如果A在地上流血并央求B报警救助,B在那里欣赏着A流血而死的惨状而不报警或者送医院救治导致A流血过多死亡的话的话,这个场合下应当承认先前的作为和后来的不作为共同成立了防卫过当。
\item A欲强奸B,C从背后一板砖拍死A。

正当防卫可以由第三方对加害者实施,不属于防卫过当,不负刑事和民事责任。
\item A强奸完毕B,B趁其转身捅死A。

A强奸B后转身离开,则B构成故意杀人。因为正当防卫针对的主体必须是正在进行不法侵害的加害者。

如果强奸完了企图继续进行绑架、勒索、杀人等行为则仍属于正当防卫。
\end{itemize}
\subsubsection{通俗解释}
简单来说,正当防卫的条件是:必须在面临\textbf{正在发生}的有可能造成重大人身财产伤亡的事件中,\textbf{一击杀死}加害者或在\textbf{加害者无法继续进行侵害后停止继续攻击加害者}才算是正当防卫。不法侵害停止后,再试图攻击加害者则属于防卫过当或故意杀人。

同时,如果加害者赤手空拳或拿塑胶棍等明显无法致死的武器进行加害,则不可以使用利刃、火器等远超加害者攻击能力的武器进行还击。
\subsection{猥亵罪}
\subsection{拐卖妇女、儿童罪}
\subsection{非法搜查罪}
\subsection{虐待家庭成员罪}
\subsection{故意伤害罪}
\subsection{走私、贩卖、运输、制造毒品罪}
\section{《民法通则》}
\section{《治安管理处罚法》}
\section{《劳动法》}
\section{《专利法》}
\section{《宗教事务条例》}
\section{《禁毒法》}
\end{document} 