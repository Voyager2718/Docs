%!Mode::"TeX:UTF-8"
\documentclass[utf-8,10pt]{ctexart}
\usepackage{amsmath} 
\usepackage{xltxtra}
\usepackage{mflogo,texnames}
\usepackage[colorlinks,
            linkcolor=black,
            urlcolor=black
            ]{hyperref}
\usepackage{fancyhdr}
\usepackage[top=1in, bottom=1in, left=1in, right=1in]{geometry}
\fancyhead[OL]{常用法律法规及其解析}
\fancyhead[OCR]{}
%\usepackage{titlesec}
\title{\textbf{常用法律法规及其解析}}
\begin{document}
\setcounter{page}{0}
\maketitle
\thispagestyle{empty}
\newpage
\pagestyle{fancy}
\tableofcontents
\newpage{}
\paragraph{}
\noindent{}
\paragraph{免责声明}
本文不为任何行动提供担保责任。也不对任何用户使用阅读本文后实施的一切行动所造成的任何后果负责。
\section{《刑法》}
\subsection{第二十条(正当防卫)}
为了使国家、公共利益、本人或者他人的人身、财产和其他权利免受正在进行的不法侵害,而采取的制止不法侵害的行为,对不法侵害人造成损害的,属于正当防卫,不负刑事责任\footnote{
正当防卫五要素:
\begin{enumerate}
\item 必须是为了使国家、公共利益、本人或者他人的人身、财产权利和其他权利免受不法侵害而实施的。
\item 必须有不法侵害行为发生。
\item 必须是正在进行的不法侵害。
\item 必须是针对不法侵害者本人实行。
\item 不能明显超过必要限度造成重大损害。
\end{enumerate}
}。

正当防卫\footnote{防卫挑拨:


防卫挑拨又叫挑拨防卫,是不法防卫行为的一种。这是指以挑拨寻衅等不正当手段,故意激怒对方,引诱对方对自己进行侵害,然后以“正当防卫”为借口,实行加害的行为。表面上,防卫挑拨具有防卫性,实质上是一种特殊形式的故意犯罪行为,故称之为防卫挑拨或者挑拨防卫。与正当防卫相比,防卫挑拨具有如下基本特征:
\begin{enumerate}
\item 行为人主观上有加害他人的犯罪意图。这是防卫挑拨与正当防卫相区别的根本特征。
\item 客观上有挑逗他人的语言、行动。防卫挑拨所反击的侵害,是由防卫挑拨行为人有意识的挑起的,没有防卫挑拨行为人的挑逗,不会有不法侵害。这是防卫挑拨最显著的特征。
\item 行为人有预谋。由于防卫挑拨需要借用「防卫」的形式,因而行为人往往是经过周密考虑、认真准备才付诸实施的。
\end{enumerate}

}明显超过必要限度造成重大损害的,应当负刑事责任,但是应当减轻或者免除处罚。

对正在进行行凶\footnote{参见《刑法》第二百三十二条 (故意杀人罪)}、杀人、抢劫\footnote{参见《刑法》第二百六十三条(抢劫罪)}、强奸\footnote{参见《刑法》第二百三十六条(强奸罪)}、绑架\footnote{参见《刑法》第二百三十九条(绑架罪)}以及其他严重危及人身安全的暴力犯罪,采取防卫行为,造成不法侵害人伤亡的,不属于防卫过当,不负刑事责任。
\subsection{第二十一条(紧急避险)}
为了使国家、公共利益、本人或者他人的人身、财产和其他权利免受正在发生的危险,不得已采取的紧急避险行为,造成损害的,不负刑事责任。

紧急避险超过必要限度造成不应有的损害的,应当负刑事责任,但是应当减轻或者免除处罚。

第一款中关于避免本人危险的规定,不适用于职务上、业务上负有特定责任的人。
\subsection{第二百三十二条(故意杀人罪)}
故意杀人的,处死刑、无期徒刑或者十年以上有期徒刑;情节较轻的,处三年以上十年以下有期徒刑。
\subsection{第二百三十三条(过失致他人死亡罪)}
过失致人死亡的,处三年以上七年以下有期徒刑;情节较轻的,处三年以下有期徒刑。

本法另有规定的,依照规定。
\subsection{第二百三十四条(故意伤害罪)}
故意伤害他人身体的,处三年以下有期徒刑、拘役或者管制。

犯前款罪,致人重伤的,处三年以上十年以下有期徒刑;致人死亡或者以特别残忍手段致人重伤造成严重残疾的,处十年以上有期徒刑、无期徒刑或者死刑。本法另有规定的,依照规定。
\subsection{第二百三十五条(过失伤害罪)}
过失伤害他人致人重伤的,处三年以下有期徒刑或者拘役。

本法另有规定的,依照规定。
\subsection{第二百三十六条(强奸罪)}
以暴力、胁迫或者其他手段强奸妇女的,处三年以上十年以下有期徒刑。

奸淫不满十四周岁的幼女的,以强奸论,从重处罚。

强奸妇女、奸淫幼女,有下列情形之一的,处十年以上有期徒刑、无期徒刑或者死刑:

(一)强奸妇女、奸淫幼女情节恶劣的;

(二)强奸妇女、奸淫幼女多人的;

(三)在公共场所当众强奸妇女的;

(四)二人以上轮奸的;

(五)致使被害人重伤、死亡或者造成其他严重后果的。
\subsection{第二百三十七条(猥亵罪)}
以暴力、胁迫或者其他方法强制猥亵妇女或者侮辱妇女的,处五年以下有期徒刑或者拘役。

聚众或者在公共场所当众犯前款罪的,处五年以上有期徒刑。

猥亵儿童的,依照前两款的规定从重处罚。
\subsection{第二百三十八条(非法拘禁罪)}
非法拘禁他人或者以其他方法非法剥夺他人人身自由的,处三年以下有期徒刑、拘役、管制或者剥夺政治权利。具有殴打、侮辱情节的,从重处罚。

犯前款罪,致人重伤的,处三年以上十年以下有期徒刑;致人死亡的,处十年以上有期徒刑。使用暴力致人伤残、死亡的,依照本法第二百三十四条、第二百三十二条的规定定罪处罚。

为索取债务非法扣押、拘禁他人的,依照前两款的规定处罚。

国家机关工作人员利用职权犯前三款罪的,依照前三款的规定从重处罚。
\subsection{第二百三十九条(绑架罪)}
以勒索财物为目的绑架他人的,或者绑架他人作为人质的,处十年以上有期徒刑或者无期徒刑,并处罚金或者没收财产;致使被绑架人死亡或者杀害被绑架人的,处死刑,并处没收财产。

以勒索财物为目的偷盗婴幼儿的,依照前款的规定处罚。
\subsection{第二百四十条(拐卖妇女、儿童罪)}
拐卖妇女、儿童的,处五年以上十年以下有期徒刑,并处罚金;有下列情形之一的,处十年以上有期徒刑或者无期徒刑,并处罚金或者没收财产;情节特别严重的,处死刑,并处没收财产:

(一)拐卖妇女、儿童集团的首要分子;

(二)拐卖妇女、儿童三人以上的;

(三)奸淫被拐卖的妇女的;

(四)诱骗、强迫被拐卖的妇女卖淫或者将被拐卖的妇女卖给他人迫使其卖淫的;

(五)以出卖为目的,使用暴力、胁迫或者麻醉方法绑架妇女、儿童的;

(六)以出卖为目的,偷盗婴幼儿的;

(七)造成被拐卖的妇女、儿童或者其亲属重伤、死亡或者其他严重后果的;

(八)将妇女、儿童卖往境外的。

拐卖妇女、儿童是指以出卖为目的,有拐骗、绑架、收买、贩卖、接送、中转妇女、儿童的行为之一的。
\subsection{第二百四十一条(收买被拐卖妇女、儿童罪)}
收买被拐卖的妇女、儿童的,处三年以下有期徒刑、拘役或者管制。

收买被拐卖的妇女,强行与其发生性关系的,依照本法第二百三十六条的规定定罪处罚。

收买被拐卖的妇女,儿童,非法剥夺、限制其人身自由或者有伤害、侮辱等犯罪行为的,依照本法的有关规定定罪处罚。

收买被拐卖的妇女,儿童,并有第二款、第三款规定的犯罪行为的,依照数罪并罚的规定处罚。

收买被拐卖的妇女,儿童又出卖的,依照本法第二百四十条的规定定罪处罚。

收买被拐卖的妇女,儿童,按照被买妇女的意愿,不阻碍其返回原居住地的,对被买儿童没有虐待行为,不阻碍对其进行解救的,可以不追究刑事责任。
\subsection{第二百四十二条(聚众阻碍解救被收买的妇女、儿童罪)}
以暴力、威胁方法阻碍国家机关工作人员解救被收买的妇女、儿童的,依照本法第二百七十七条的规定定罪处罚。

聚众阻碍国家机关工作人员解救被收买的妇女、儿童的首要分子,处五年以下有期徒刑或者拘役;其他参与者使用暴力、威胁方法的,依照前款的规定处罚。
\subsection{第二百四十三条(诬告陷害罪)}
捏造事实诬告陷害他人\footnote{
本罪在客观上表现为捏造他人犯罪的事实,向国家机关或有关单位告发,或者采取其他方法足以引起司法机关的追究活动。

首先,必须捏造犯罪事实,即无中生有、栽赃陷害、借题发挥把杜撰的或他人的犯罪事实强加于被害人。所捏造的犯罪事实,只要足以引起司法机关追究被害人的刑事责任即可,并不要求捏造详细情节与证据。

其次,必须向国家机关或有关单位告发,或者采取其他方法足以引起司法机关的追究活动,告发方式多种多样,如口头的、书面的、署名的、匿名的、直接的、间接的等等。如果只捏造犯罪事实,既不告发,也不采取其他方法引起司法机关追究的,则不构成本罪。

再次,必须有特定的对象。如果没有特定对象,就不可能导致司法机关追究某人的刑事责任,因而不会侵犯他人的人身权利。当然,特定对象并不要求行为人点名道姓,只要告发的内容足以使司法机关确认对象是谁就构成诬告陷害罪。至于被诬陷的对象是遵纪守法的公民,还是正在服刑的犯人,以及是否因被诬告而受到刑事处分,均不影响本罪的成立。诬陷没有达到法定年龄或者没有辨认或控制能力的人犯罪,属于对象不能犯,仍构成诬告陷害罪。
},意图使他人受刑事追究,情节严重的,处三年以下有期徒刑、拘役或者管制;造成严重后果的,处三年以上十年以下有期徒刑。

国家机关工作人员犯前款罪的,从重处罚。

不是有意诬陷,而是错告,或者检举失实的,不适用前两款的规定。
\subsection{第二百六十三条(抢劫罪)}
以暴力、胁迫或者其他方法抢劫财物的\footnote{
抢劫罪的界定:
\begin{enumerate}
\item 侵犯的客体是复杂客体,即既侵犯公私财产所有权,又侵犯了公民的人身权利,这是本罪区别于其他侵犯财产罪的重要特征。
\item 客观方面表现为对公私财物的所有人、保管人或者守护人当场使用暴力、胁迫或者其他方法,抢走财物或者迫使被害人交出财物的行为。

    所谓暴力,是指对被害人的身体实施暴力,例如,以捆绑、殴打、伤害、杀害等暴力行为,使被害人不敢抗拒或者不能抗拒,而当场交出财物或者抢走财物。另有的犯罪人虽然没有持有凶器,但采用突然袭击方法将被害人推倒,或者用卡脖子、拳打、脚踢等方面加害被害人,有的还造成伤害或者死亡,这些都属于暴力范围。

    所谓胁迫,是指以暴力相威胁,对被害人进行精神强制,使其产生恐惧,不敢反抗,被迫当场交出财物,或者不敢阻止犯罪人的行为而任其将财物劫走。抢劫犯罪的胁迫行为,一般是针对被害人本人实施,有时也可能是针对在场的被害人的亲属。但只要足以使被害人恐惧,而不得不当场交出财物或者任其将财物劫走,都可以构成抢劫罪。

    所谓其他方法,是指除了暴力、胁迫方法以外,采用使被害人不知反抗或者丧失反抗能力的各种方法。例如,用酒灌醉、用药物麻醉等方法,使被害人处于昏睡、不能反抗的状态,而当场将财物劫走。但是,需要明确,这里所说的昏睡、不能反抗的状态,必须是由于犯罪人的行为直接造成的。
\item 抢劫抢劫罪的犯罪主体为一般主体,已满14周岁不满16周岁的人犯抢劫罪的,应当负刑事责任。
\item 主观方面只能是故意,而且具有非法占有公私财物的目的。如果只是抢回自己被骗去或者赌输的财物,虽然使用了暴力或者胁迫方法,但不具有非法占有他人财物的目的,不构成抢劫罪。
\end{enumerate}
},处三年以上十年以下有期徒刑,并处罚金;有下列情形之一的,处十年以上有期徒刑、无期徒刑或者死刑,并处罚金或者没收财产:

(一)入户抢劫的;

(二)在公共交通工具上抢劫的;

(三)抢劫银行或者其他金融机构的;

(四)多次抢劫或者抢劫数额巨大的;

(五)抢劫致人重伤、死亡的;

(六)冒充军警人员抢劫的;

(七)持枪抢劫的;

(八)抢劫军用物资或者抢险、救灾、救济物资的。
\subsection{第二百六十四条(盗窃罪)}
盗窃公私财物,数额较大或者多次盗窃的,处三年以下有期徒刑、拘役或者管制,并处或者单处罚金;数额巨大或者有其他严重情节的,处三年以上十年以下有期徒刑,并处罚金;数额特别巨大或者有其他特别严重情节的,处十年以上有期徒刑或者无期徒刑,并处罚金或者没收财产;有下列情形之一的,处无期徒刑或者死刑,并处没收财产:

(一)盗窃金融机构,数额特别巨大的;

(二)盗窃珍贵文物,情节严重的。
\subsection{第二百四十六条(诽谤罪)}
以暴力或者其他方法公然侮辱他人或者捏造事实诽谤他人\footnote{
诽谤罪三要素:
\begin{enumerate}
\item 须有捏造和散布某种事实的行为,即诽谤他人的内容完全是虚构的。如果散布的不是凭空捏造的,而是客观存在的事实,即使有损于他人的人格、名誉,也不构成本罪。
\item 捏造事实诽谤他人的行为必须属于情节严重。虽有捏造事实诽谤他人的行为,但没有达到情节严重的程度,则不能以诽谤罪罪论处。所谓情节严重,主要是指多次捏造事实诽谤他人的;捏造事实造成他人人格、名誉严重损害的;捏造事实诽谤他人造成恶劣影响的;诽谤他人致其精神失常或导致被害人自杀的等等情况。
\item 主观上必须是故意,即明知自己所发布的内容是足以损害他人名誉的虚假事实,明知自己的行为会发生损害他人名誉的危害结果,并且希望这种结果的发生。如果嫌疑人将虚假事实误认为是真实事实加以扩散,或者把某种虚假事实进行扩散但无损害他人名誉的目的,则不构成诽谤罪。
\end{enumerate}
},情节严重的,处三年以下有期徒刑、拘役、管制或者剥夺政治权利。

前款罪,告诉的才处理,但是严重危害社会秩序和国家利益的除外。
\subsection{第二百六十七条(抢夺罪)}
抢夺公私财物\footnote{
抢夺罪的界定:

抢夺罪,是指以非法占有为目的,乘人不备,公然夺取数额较大的公私财物的行为。本罪的主要特征:
\begin{enumerate}
\item 本罪侵犯的客体,是公私财物的所有权。其对象是公私财物。
\item 本罪在客观方面,表现为乘人不备,公然夺取公私财物的行为。这里所说的公然夺取,包括两层意思:一是指行为人当着公私财物所有者或者保管者的面,乘其不备,公开夺取其财物。公开夺取是本罪区别于盗窃罪的秘密窃取的一个重要标志。二是指行为人闯入他人住宅,面对房屋主人夺走桌上电视机、收录机等财物,或者深夜在僻静小巷内抢走一个行人手中的财物,虽无他人在场,也是公然夺取。
\item 本罪为一般主体,即凡是已年满16周岁,并具有刑事责任能力的人,都可成为本罪的主体。
\item 本罪在主观方面只能是故意,并以非法占有公私财物为目的。
\end{enumerate}
},数额较大的,处三年以下有期徒刑、拘役或者管制,并处或者单处罚金;数额巨大或者有其他严重情节的,处三年以上十年以下有期徒刑,并处罚金;数额特别巨大或者有其他特别严重情节的,处十年以上有期徒刑或者无期徒刑,并处罚金或者没收财产。

携带凶器抢夺的,依照本法第二百六十三条的规定定罪处罚。
\subsection{第二百七十九条(冒充国家机关工作人员招摇撞骗罪)}
冒充国家机关工作人员招摇撞骗的,处三年以下有期徒刑、拘役、管制或者剥夺政治权利;情节严重的,处三年以上十年以下有期徒刑。

冒充人民警察招摇撞骗的,依照前款的规定从重处罚。
\subsection{第二百八十条(伪造、变造、买卖国家机关公文、证件、印章罪)}

伪造、变造、买卖或者盗窃、抢夺、毁灭国家机关的公文、证件、印章的,处三年以下有期徒刑、拘役、管制或者剥夺政治权利;情节严重的,处三年以上十年以下有期徒刑。

伪造公司、企业、事业单位、人民团体的印章的,处三年以下有期徒刑、拘役、管制或者剥夺政治权利。

伪造、变造居民身份证的,处三年以下有期徒刑、拘役、管制或者剥夺政治权利;情节严重的,处三年以上七年以下有期徒刑。
\subsection{第二百九十三条(寻衅滋事罪)}
有下列寻衅滋事行为之一,破坏社会秩序的,处五年以下有期徒刑、拘役或者管制:

(一)随意殴打他人,情节恶劣的;

(二)追逐、拦截、辱骂他人,情节恶劣的;

(三)强拿硬要或者任意损毁、占用公私财物,情节严重的;

(四)在公共场所起哄闹事,造成公共秩序严重混乱的。
\section{《民法通则》}
\subsection{第一百二十八条(正当防卫)}
因正当防卫造成损害的,不承担民事责任。正当防卫超过必要的限度,造成不应有的损害的,应当承担适当的民事责任。
\subsection{第一百二十九条(紧急避险)}
因紧急避险造成损害的,由引起险情发生的人承担民事责任。如果危险是由自然原因引起的,紧急避险人不承担民事责任或者承担适当的民事责任。因紧急避险采取措施不当或者超过必要的限度,造成不应有的损害的,紧急避险人应当承担适当的民事责任。
\section{《治安管理处罚法》}
\subsection{第六十六条(卖淫、嫖娼)}
卖淫、嫖娼的,处十日以上十五日以下拘留,可以并处五千元以下罚款;情节较轻的,处五日以下拘留或者五百元以下罚款。

在公共场所拉客招嫖的,处五日以下拘留或者五百元以下罚款。
\subsection{第六十七条(引诱、容留、介绍他人卖淫)}
引诱、容留、介绍他人卖淫的,处十日以上十五日以下拘留,可以并处五千元以下罚款;情节较轻的,处五日以下拘留或者五百元以下罚款。
\subsection{第七十三条(教唆、引诱、欺骗他人吸食、注射毒品)}
教唆、引诱、欺骗他人吸食、注射毒品的,处十日以上十五日以下拘留,并处五百元以上二千元以下罚款。
\section{《侵权责任法》}
\subsection{第三十条(正当防卫)}
因正当防卫造成损害的,不承担责任。正当防卫超过必要的限度,造成不应有的损害的,正当防卫人应当承担适当的责任。
\subsection{第三十一条(紧急避险)}
因紧急避险造成损害的,由引起险情发生的人承担责任。如果危险是由自然原因引起的,紧急避险人不承担责任或者给予适当补偿。紧急避险采取措施不当或者超过必要的限度,造成不应有的损害的,紧急避险人应当承担适当的责任。
\subsection{第三十七条(安全保障义务)}
宾馆、商场、银行、车站、娱乐场所等公共场所的管理人或者群众性活动的组织者,未尽到安全保障义务,造成他人损害的,应当承担侵权责任。

因第三人的行为造成他人损害的,由第三人承担侵权责任;管理人或者组织者未尽到安全保障义务的,承担相应的补充责任。
\section{《人民警察使用警械和武器条例》}
\subsection{第三条(警械与武器)}
第三条本条例所称警械,是指人民警察按照规定装备的警棍、催泪弹、高压水枪、特种防暴枪、手铐、脚镣、警绳等警用器械;所称武器,是指人民警察按照规定装备的枪支、弹药等致命性警用武器。
\subsection{第七条(警械的使用)}
人民警察遇有下列情形之一,经警告无效的,可以使用警棍、催泪弹、高压水枪、特种防暴枪等驱逐性、制服性警械:

(一)结伙斗殴、殴打他人、寻衅滋事、侮辱妇女或者进行其他流氓活动的;

(二)聚众扰乱车站、码头、民用航空站、运动场等公共场所秩序的;

(三)非法举行集会、游行、示威的;

(四)强行冲越人民警察为履行职责设置的警戒线的;

(五)以暴力方法抗拒或者阻碍人民警察依法履行职责的;

(六)袭击人民警察的;

(七)危害公共安全、社会秩序和公民人身安全的其他行为,需要当场制止的;

(八)法律、行政法规规定可以使用警械的其他情形。

人民警察依照前款规定使用警械,应当以制止违法犯罪行为为限度;当违法犯罪行为得到制止时,应当立即停止使用。
\subsection{第八条(武器的使用)}
人民警察判明有下列暴力犯罪行为的紧急情形之一,经警告无效的,可以使用武器:

(一)放火、决水、爆炸等严重危害公共安全的;

(二)劫持航空器、船舰、火车、机动车或者驾驶车、船等机动交通工具,故意危害公共安全的;

(三)抢夺、抢劫枪支弹药、爆炸、剧毒等危险物品,严重危害公共安全的;

(四)使用枪支、爆炸、剧毒等危险物品实施犯罪或者以使用枪支、爆炸、剧毒等危险物品相威胁实施犯罪的;

(五)破坏军事、通讯、交通、能源、防险等重要设施,足以对公共安全造成严重、紧迫危险的;

(六)实施凶杀、劫持人质等暴力行为,危及公民生命安全的;

(七)国家规定的警卫、守卫、警戒的对象和目标受到暴力袭击、破坏或者有受到暴力袭击、破坏的紧迫危险的;

(八)结伙抢劫或者持械抢劫公私财物的;

(九)聚众械斗、暴乱等严重破坏社会治安秩序,用其他方法不能制止的;

(十)以暴力方法抗拒或者阻碍人民警察依法履行职责或者暴力袭击人民警察,危及人民警察生命安全的;

(十一)在押人犯、罪犯聚众骚乱、暴乱、行凶或者脱逃的;

(十二)劫夺在押人犯、罪犯的;

(十三)实施放火、决水、爆炸、凶杀、抢劫或者其他严重暴力犯罪行为后拒捕、逃跑的;

(十四)犯罪分子携带枪支、爆炸、剧毒等危险物品拒捕、逃跑的;

(十五)法律、行政法规规定可以使用武器的其他情形。

人民警察依照前款规定使用武器,来不及警告或者警告后可能导致更为严重危害后果的,可以直接使用武器。

第十条人民警察遇有下列情形之一的,不得使用武器:

(一)发现实施犯罪的人为怀孕妇女、儿童的,但是使用枪支、爆炸、剧毒等危险物品实施暴力犯罪的除外;

(二)犯罪分子处于群众聚集的场所或者存放大量易燃、易爆、剧毒、放射性等危险物品的场所的,但是不使用武器予以制止,将发生更为严重危害后果的除外。
\subsection{第十一条(停止使用武器)}
人民警察遇有下列情形之一的,应当立即停止使用武器:

(一)犯罪分子停止实施犯罪,服从人民警察命令的;

(二)犯罪分子失去继续实施犯罪能力的。
\section{《宗教事务条例》}
\subsection{第二十条}
宗教活动场所可以按照宗教习惯接受公民的捐献,但不得强迫或者摊派。

非宗教团体、非宗教活动场所不得组织、举行宗教活动,不得接受宗教性的捐献。
\section{《关于我国社会主义时期宗教\\问题的基本观点和基本政策》}
\noindent 《关于我国社会主义时期宗教问题的基本观点和基本政策》第六节第三句:

任何人都不应当到宗教场所进行无神论的宣传,或者在信教群众中发动有神还是无神的辩论;但是任何宗教组织和教徒也不应当在宗教活动场所以外布道、传教,宣传有神论,或者散发宗教传单和其他未经政府主管部门批准出版发行的宗教书刊。
\end{document}
