%!Mode::"TeX:UTF-8"
\documentclass[UTF8,12pt]{ctexart}
\usepackage{amsmath}
\usepackage{xltxtra}
\usepackage{mflogo,texnames}
\usepackage[colorlinks,
            linkcolor=black,
            urlcolor=black
            ]{hyperref}
\usepackage{fancyhdr}
\usepackage{color}
\usepackage{listings}
\lstset{numbers=none,numberstyle=\tiny,basicstyle=\ttfamily \small,commentstyle=\color{gray},keywordstyle=\color{blue},showstringspaces=false,tabsize=4}

\title{\textbf{Rockets}\\\small{The most incredible artificiality in the world!}}
\author{YANG Zhipeng}
\date{}
\begin{document}
\maketitle
\pagestyle{fancy}
\section{Momentum Theorem}
In our world, we have an universal theorem, which named the Momentum Theorem.

According to this theorem, when someone push the other one, he will also receive a force from whom he pushed.

Futhermore, we have this formula to explain the momentum.

\begin{align}
m_{1}v_{1}&=m_{2}v_{2}\\
Ft&=mv
\end{align}
When the rocket engine thrust. There will be a force pushing the rockets go up.

For example, if a rocket thrust 100kg fuel per second with a speed of $1000m/s$, according to the formula mentioned above, we have:
\begin{align}
F&=\frac{mv}{t}\\
F&=\frac{100kg\times1000m/s}{1s}\\
F&=100000N
\end{align}
So we can send a rocket to space by using these princeples.
\section{Orbits}
But, to pin a rocket to the space, we need to consider a little more.

It's the orbits.

Why can a rocket go around the earth, but not falling down? The answer is both simple and complex. It's the speed.

It's the time for formulas.
\begin{align}
F&=G\frac{m_{1}m_{2}}{r^{2}}\\
F&=m_{2}\frac{v^{2}}{r}
\end{align}
G is Gravitational Constant, which is $6.67\times10^{-11}$. $m_{1}$ is the weight of our earth. $m_{2}$ is the weight of the satellite. And $r$ is the radius of our earth.

So, after a little calculation, we have:
\begin{align}
F=G\frac{m_{1}m_{2}}{r^{2}}&=m_{2}\frac{v^{2}}{r}\\
G\frac{m_{1}}{r}&=v^{2}\\
v&=\sqrt{G\frac{m_{1}}{r}}\\
v&\approx7907.13m/s
\end{align}
It means, if a satellite take off with a speed of 7907.13m/s, it can automatically go around the earth. $7907.13m/s$ is also the first cosmic velocity.
\section{Escape Velocity}
We have 3 cosmic velocities, which named 1st cosmic velocity, 2nd cosmic velocity, 3rd cosmic velocity.

A satellite can lift-off with the speed of the 1st cosmic velocity.

If a satellite wants to leave our earth, it has to have the 2nd cosmic velocity which is $11.2km/s$.

If it wants to go out of our solar system, it is going to have the 3rd cosmic velocity which is $16.7km/s$.
\section{Age Of Discovery}
In my opinion, we are going to have the new Age of Discovery in decades.

Just like hundreds of years ago, we invented a ships which can travel all over the world. So we have discovered the world.

Whatmore in noadays, we have a lot of rockets, and we have invented a lot of technologies that will help us travelling around our solar system.

So, in my view point, we are going to have a new Age of Discovery.
\section{References}
\noindent
\url{http://en.wikipedia.org/wiki/Momentum}\\
\url{http://en.wikipedia.org/wiki/Saturn_V}\\
\url{http://en.wikipedia.org/wiki/Escape_velocity}\\
\url{http://www.pep.com.cn/}
\paragraph{}
\rightline{Powered By \LaTeX{}}
\end{document} 