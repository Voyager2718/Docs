%!Mode::"TeX:UTF-8"
\documentclass[UTF8,twocolumn]{ctexart}
\usepackage{xltxtra}
\usepackage{mflogo,texnames}
\usepackage[colorlinks,
            linkcolor=black,
            urlcolor=black
            ]{hyperref}
\usepackage{listings}
\lstset{numbers=none,numberstyle=\tiny,basicstyle=\ttfamily
\small,commentstyle=\color{gray},keywordstyle=\color{blue},
showstringspaces=false,tabsize=4}
\title{Ocaml手记}
\author{杨志鹏}
\date{}
\begin{document}
\maketitle
\paragraph{}
本文档使用PCL协议:\url{http://voyager2718.github.io/PCL.pdf}
\\\\
\section{环境}
\begin{itemize}
\item open **;; 打开**模块

\item \#{}use "**";; 打开**文件

\item ocamlc -c ** 在cmd/terminal里面, 将**文件编译

\item 打开一个自定义模块的时候, 需要在cmd/terminal使用以下语法:ocaml ***.cmo(意思是:使用ocaml, 并同时引入***.cmo)

(定义模块的文件名必须为小写, 但是open的时候必须首字母大写)

\item OCaml变量首字母不能大写\\
 首字母大写的是只能是某type类型的元素

\item let main () = .. ;; let \underline{ } = main();; 通过let \underline{ } = main()定义main是主函数.

\item 先用ocamlc -c将程序编译出来, 然后用ocamlc -o ** *** ****将程序编译成可执行文件. (其中*为输出文件名, **为依赖文件, ***为源代码(.ml))

\item 带参数运行的方法:./* arg1 arg2 (其中*为可执行文件, arg1, arg2为参数)
\end{itemize}
\section{基础知识}
\begin{lstlisting}[language=C++]
let (a,b) = (1,"2");;
(* 将 a , b 的值设置成1和"2"(分别是 int 和 string ) *)

let d = (1,2);;
(* 将 d 的值设置成(1,2)( d 的类型是 int*int) *)

fst d;;
(* 将输出 d 中第一个“元素”, 即1 *)

snd d;;
(* 将输出 d 中第二个“元素”, 即2 *)

for i = 10 to 0 do
	print_int i;
done;;
(* 这个 for 循环的意思是 for ( int i = 10 ; i < 0 ; i++ ){}( C 系语言) *)

for i = 10 to 0 do
	print_int i;
done;;
(* 这个 for 循环的意思是 for ( int i = 10 ; i < 0 ; i++ ){}( C 系语言) *)

type complex = { r: float; i: float};;
(* 定义了一个复数的类型. 其中 r 和 i 均是float *)

let c = {r= 1.; i= 2.};;
(* 定义了一个 complex 类型的 c , 并将其值设成 1(float)和2(float) *)

let add_c x y =
{
	r=x.r+.y.r;
	i=x.i+.y.i;
};;
(* 定义了一个函数add_c, 将两个complex类型相加并返回 *)

let add_c x y = match (x,y) with
	({r=x1;i=y1},{r=x2;i=y2})
->{r=x1+.x2;i=x1+.x2};;
(* 上一个函数, 使用 match..with 的版本 *)

let c = add_c c c;;
(* 将 c 作为参数, 返回 2c *)

let c2 = { c with r = 0.};;
(* 定义了一个 complex 类型的 c1, r=0(float), 其它元素与 c 中的元素相同 *)

Sys.argv;;
(* 保存运行参数的 string 数组 *)
\end{lstlisting}
\section{Big\underline{ }int}
\begin{lstlisting}[language=CAML]
let a = 123;
\end{lstlisting}
Powered by \XeLaTeX{}
\end{document} 