%!Mode::"TeX:UTF-8"
\documentclass[UTF8]{ctexart}
\usepackage[colorlinks,urlcolor=black,linkcolor=black]{hyperref}
\usepackage{fancyhdr}
\usepackage{enumerate}
\title{我的参数}
\author{杨志鹏}
\begin{document}
\maketitle
\pagestyle{fancy}
\tableofcontents{}
\newpage
\paragraph{}
\begin{center}
(以下所有项目如非特别附注,全部忽略重要性先后)
\end{center}
\section{心情}
\subsection{喜欢的事情}
\begin{itemize}
\item 被别人夸聪明
\item 辛苦做好一个作品被称赞
\item 讨论政治/历史
\item 看知乎/果壳/百科/军事
\item 新奇有趣的东西
\item 先进的科技
\item 有动感的东西(如飞机/发动机/汽车等)
\item 精密而且有机械感的东西\ 规整的东西
\end{itemize}
\subsection{喜欢的感觉}
\begin{itemize}
\item 被重视的感觉
\item 被依赖的感觉
\item 被信任的感觉
\item 热烈讨论的氛围
\end{itemize}
\subsection{讨厌的事情(排名分先后)}
\begin{itemize}
\item 不被在乎
\item 参杂利益的感情
\item 将自己弱点告诉别人,却被别人利用来攻击自己。
\item 不公平
\item 辛辛苦苦做好一个自己很喜欢的东西,不但不被鼓励,还被打击。
\item 被人以不当理由不听取我的意见(例如:年龄/资历等)。
\item 凭借自己主观判断一些事情且不给别人理论的机会。
\item 用“你懂的”或类似词语暗示一些未经证实和没有确定证据的谣言。
\item 故作玄虚
\item 说话说一半
\end{itemize}
\subsection{身体上的怪癖}\begin{itemize}
\item 喜欢汽油的味道
\item 喜欢闻自己脱下来的臭袜子
\end{itemize}
\subsection{精神上的怪癖}
\begin{itemize}
\item 喜欢先抑后扬,先假意要做出一些不好的事情,然后待对方感到受挫的时候再给予对方想要的东西,满足对方的需求。因为感觉这样获得的喜悦更深更强。
\item 爱吐槽,但是其实很多时候只是想吐槽而已,并非真的这样想。尤其是丑/胸小/矮之类先天性特征。
\end{itemize}
\section{价值观、人生观、世界观}
\begin{itemize}
\item 要不不要,若要就要最好的。对认为重要的东西如果得不到最好的宁愿不要。如果确实得不到最好,那就无论是中还是下都可以接受(但是此时已经破罐子破摔)。
\item 诚信极其重要。无论是说出还是做出不守信的行为都会被在心中记录在案。最近也在刻意规范自己在诚信方面的行为:遇到类似于承诺的话会
三思而后行。错误许下承诺也会履约。
\item 非常喜欢和仰慕有知识有文化的人。
\item 一个人爱与他血脉相连的国家和民族是理所当然的事情。只要种族和国家还存在,任何人都应该为其国家和民族感到由衷的自豪和发自内心的喜爱。
\item 基本不会发脾气,如果触犯认为非常重要的原则(如诚信)或多次触犯认为重要的原则则会减分至“0”,并消失在那个人的生活中。
\item 不喜欢拍很多次拖。极度希望以尽可能少的拍拖次数结婚。因为认为爱的原始冲动会随着次数的增加而陡然下降。
\item 懒得管的事情会直接装傻。
\item 不爽社会上很多的出轨和虐待伴侣的行为,并且一直在提醒自己:如果找到两情相悦的真爱并能结婚,一定要让她成为世界上最幸福的女人。
\item 如果碰到严重犯罪行为,一定要以最严厉的方法惩罚。以儆效尤,杀鸡儆猴。
\item 反对卖淫、精神类药物、买卖器官等物化人类的行为合法化。
\item 不喜欢中国被其他国家和民族移民大量涌入。汉族是最优秀的民族(之一),我们能创造辉煌的文明。
\item 为了创造一个完美、平等、和谐的世界,任何有能力的人都应该按以下顺序帮助其他人:本地人→本地域人→本国人→本肤色的人→全世界。
\item 当自己有能力的时候,需要去帮助那些暂时没有能力的人,且教会他们如何做一个更加有用的人。
\item 做人不能蓄意做会伤害到他人的事情,且尽可能去考虑他人的处境和心情,以防伤害到对方。
\item 不该从立法上删除死刑,但是可以在司法上不执行死刑。
\item 保守主义者,反对婚前性行为(但订婚后或事实婚姻等除外),支持一夫一妻,支持男人婚后出轨支付大量抚养费以保障妻儿原有生活水平。
\item 相信国与国之间是一个完全竞争的丛林社会。
\end{itemize}
\section{梦想}
\begin{itemize}
\item 让中国和中华民族成为地球上最伟大的国家和民族。
\item 让中国土地上再也没有受压迫的人。
\item 成为超级有钱的人,然后做一个在教育、平等、和平等方面影响巨大的慈善机构。
\item 若能和心爱的人结婚,希望死后骨灰能混合在一起被送入绕日轨道。
\item 成为一个举国瞩目且被评价为“最接近神的伟大男人,最完美的男人”的男人。
\end{itemize}
\section{行为}
\subsection{兴趣}
\begin{itemize}
\item 刷知乎/果壳/百科/军事类网站
\item 吃好吃的
\item 了解历史类的东西
\item 做模型
\item 做喜欢的菜式(只是爱尝试)
\item 用电脑做各种不同有趣的东西。
\item 刻章、绘画(不好看)、耍乐器、写毛笔字。
\end{itemize}
\subsection{喜欢的行为}
\begin{itemize}
\item 游泳
\item 羽毛球
\item 坐在车上望着车外
\item 射击/射箭/Wargames
\end{itemize}
\section{信仰}
信仰不可知论。认为即使“神”存在,也不可能是一个人格化的“上帝”,更不会理茫茫宇宙中那卑微,普通的我们。

退一万步说,即使人格化的“上帝”存在,我们不应该对其卑躬屈膝,成为其仆人和奴隶。

认为所有一神论、全能神论的宗教都应该被解散。
\section{派系}
左派右倾。赞同应该给最底层的人、无能力照顾自己的人给予最低限度的生命权,也支持竞争和精英社会。
\section{最喜欢的…}
\subsection{帝王}
\begin{itemize}
\item 李世民 开创中国最繁荣、开放、自由、文明的封建帝国。让世人知道什么是统治的艺术。
\item 秦始皇 开创了中国统一的局面,奠定了中国千年文明的根基。
\end{itemize}
\subsection{食物}
\begin{itemize}
\item 干蒸
\item 虾饺
\item 猪肝肠粉
\item 牛肉肠粉
\item 馄饨面
\item 煎饺子
\item 炸馄饨
\end{itemize}
\subsection{颜色}
黑色:简单、低调、深沉、内敛。
\subsection{虚构人物}
哆啦A梦:第一个喜欢的卡通人物,第一个有印象的卡通人物,第一本买的漫画。
\section{优点}
\begin{itemize}
\item 会很关心认为重要的人
\item 对认为重要的人,相对较会换位思考,思索其内心
\item 对各种人都会给予尊重
\item 不会讽刺他人
\item 很容易忘记使自己生气的事情
\item 被指出真正的缺点不会恼怒
\item 对认为重要的事情很认真
\item 相对不会凭外貌等外部条件判断一个人的内心和价值
\end{itemize}
\section{缺点}
\begin{itemize}
\item 遇到挫折就想推倒重来
\item 贪小便宜——捡到钱绝对不会还回去的类型
\item 喜欢用别人不懂的知识装B
\item 懒惰
\item 好为人师
\item 经常会忘记事情
\item 会在不应该认真的场合认真
\item 某些时候容易不自信
\end{itemize}
\section{目前烦恼}
\begin{itemize}
\item 希望自己能成为一个强大且温柔的人。
\item 找不到/得不到心爱的且是心灵契合的Soulmate作为伴侣。
\item 工作后不能很快地赚回留学用的钱。
\end{itemize}
\section{希望成为…}
\begin{itemize}
\item 会真心地为他人的不幸感到伤心,为他人的成功而感到开心
\item 讲信用、诚信,忠诚、专一;言出必行,一诺千金
\item 谦逊,尊重他人,懂得考虑他人感受
\item 强大且温柔,懂得尊重他人(的感受、利益、选择等)
\item 以适当的途径去伸张正义
\item 会以正确的方式关怀他人
\item 不卑不亢,不示弱,不欺小
\item 通晓中国古代文化,能诗能文,能笔能墨
\item 兴趣广泛,对其中一部分专业
\item 通晓天文、物理、生物、东西音乐、历史、政治
\end{itemize}
的人
\section{喜欢的类型(男)}
\noindent (排名分先后)\\
那些:
\begin{enumerate}[1.]
\item 会真心地为他人的不幸感到伤心,为他人的成功而感到开心
\item 纯粹,不会想去害人和用功利心去做事;

善良,能考虑对方感受,不忍伤害他人
\item 不自大,不自卑,耐心听取他人的意见,不强迫他人
\item 诚实却不会出言伤人
\item 成熟而不世故
\item 聪明
\item 喜欢思考、学习
\item 爱反省自身过错
\item 乐于助人
\item 兴趣广泛且有专精
\end{enumerate}
的人
\section{喜欢的类型(女)}
\noindent (排名分先后)\\
那些:
\begin{enumerate}[1.]
\item 会真心地为他人的不幸感到伤心,为他人的成功而感到开心
\item 单纯却不愚蠢,不会想去害人和用功利心去做事;

善良,能考虑对方感受,不忍伤害他人
\item 诚实却不会出言伤人
\item 聪明
\item 喜欢思考、学习
\item 活泼、阳光
\item 兴趣广泛且有专精
\end{enumerate}
的人
\paragraph{}
\section{用词标准}
\begin{itemize}
\item ..:有点无语
\item …:无语/犹豫
\item ……/………/…………:很无语
\item \~{}:有点得意
\item \~{}\~{}/\~{}\~{}\~{}/\~{}\~{}\~{}\~{}:很得意
\item ..\~{}:无语却又觉得挺有趣
\item 在对话中突然加句号:严肃、逐字逐句说
\item 喜恶度(从喜到恶):爱-很喜欢-喜欢-有兴趣-还好-一般-一般般-不喜欢-讨厌-反感-厌恶-憎恶
\end{itemize}
\rightline{Powered By \TeX{}}
\end{document} 