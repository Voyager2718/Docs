%!Mode::"TeX:UTF-8"
\documentclass[UTF8]{ctexart}
\usepackage[left=1in, right=1in]{geometry}
\usepackage{amsmath}
\usepackage{xltxtra}
\usepackage{mflogo,texnames}
\usepackage[colorlinks,
            linkcolor=black,
            urlcolor=black
            ]{hyperref}
\usepackage{fancyhdr}
\usepackage{color}
\usepackage[utf8]{inputenc}%a essayer
\usepackage{listings}
\lstset{numbers=none,numberstyle=\tiny,basicstyle=\ttfamily
\small,commentstyle=\color{gray},keywordstyle=\color{blue},
showstringspaces=false,tabsize=4}

\title{编程手记}
\author{杨志鹏}
\begin{document}
\maketitle
\pagestyle{fancy}
\tableofcontents
\newpage
\section{C/C++}
\subsection{类}
\begin{itemize}
\item {\LARGE 虚函数}

    在基类中声明虚函数的话,用基类的指针指向基类的对象的时候会调用基类的函数,指向派生类的对象的时候会调用派生类的函数。(一旦在基类中声明虚函数,派生类中相同的函数同时也成为虚函数)

虚函数在基类中使用virtual定义,如:
\begin{lstlisting}[language=C++]
virtual void function(){...}
\end{lstlisting}
\item {\LARGE 纯虚函数}

纯虚函数在基类中用virtual和=0定义,有纯虚函数的类都是抽象类,如:
\begin{lstlisting}[language=C++]
virtual void function()=0
\end{lstlisting}
\item {\LARGE 虚基类}

设class A,B,C,D。A中有变量a,且B、C继承A,D继承B、C。

如果有D d。此时若cout<<d.a<<endl;会出错。因为d.a可能是B继承的a,也可能是C继承的a。

所以在B、C继承A的时候,要加virtual,声明B、C用虚基类的方式继承A(必须B、C同时声明)。那么当D继承B、C的时候,B和C只保留一份a的副本。也就是说d.A::a、d.B::a、d.C::a、d.a在内存中均指向一个地址。

{\color{red}※类B、C、D中不能再定义a}

虚基类使用如下方法定义:
\begin{lstlisting}[language=C++]
class A{
public:
    int a;
};
class B: virtual public A{
    ...
};
class C: virtual public A{
    ...
};
class D: public B, public C{
    ...
};
\end{lstlisting}
\end{itemize}
\subsection{指针}
\begin{itemize}
\item {\LARGE 普通函数指针}

使用如下方法定义一个指向返回值为int*,参数为int*的函数的指针:
\begin{lstlisting}[language=C++]
int (*(*pf)(int*));
\end{lstlisting}
\item {\LARGE 参数为函数指针且返回函数指针的函数}

使用如下方法定义一个参数为函数指针且返回函数指针的函数:
\begin{lstlisting}[language=C++]
int* (*func2(int (*(*pf)(int*)),int *p))(int*){}
\end{lstlisting}

或(C++11)

\begin{lstlisting}[language=C++]
int* (*func2(int (*(*pf)(int*)),int *p))(int*){}
\end{lstlisting}
auto func3(int (*(*pf)(int*)),int *p) -> int *(*)(int*){}
\item {\LARGE 使用typedef定义函数指针}

使用如下方法定义一个指向返回值为int*,参数为int*的函数的指针:
\begin{lstlisting}[language=C++]
typedef int*(*PF)(int*);
\end{lstlisting}
\end{itemize}
\subsection{vector(C++)}
头文件:
\begin{lstlisting}[language=C++]
#include<vector>
\end{lstlisting}
\subsection{多线程}
在最新的C++11中,可以通过\#include<thread>使用多线程。
\begin{lstlisting}[language=C++]
#include<thread>
#include<iostream>
using namespace std;
void Task0(int a){
    cout<<"Task"<<a<<endl;
}
void Task1(int a){
    cout<<"Task"<<a<<endl;
}
int main(int args, char* argc[]){
    thread t0(Task0,0);
    thread t1(Task1,1);
    
    t0.join();
    t1.join();
    
    system("pause");
    return 0;
}
\end{lstlisting}
\section{C\#}
\section{Java}
\begin{itemize}
\item 按位右移,并补0(下列例子结果是000000010,如果用>>,则是0000010):

\begin{lstlisting}[language=Java]
8>>>2
\end{lstlisting}
\end{itemize}
\section{\LaTeX{}}
\begin{itemize}
\item 定义字体颜色:

\begin{lstlisting}[language=TeX]
{\color{red/blue/black..} TEXT}
\end{lstlisting}

或

\begin{lstlisting}[language=TeX]
{\color[rgb]{0.0/0.0/0.0} TEXT}
\end{lstlisting}
\end{itemize}
\section{PHP}
\section{MySQL}
\section{Javascript}
\section{Python}
\section{Ocaml}
\end{document}
