%!Mode::"TeX:UTF-8"
\documentclass[utf-8,10pt]{ctexart}
\usepackage{amsmath}
\usepackage{xltxtra}
\usepackage{mflogo,texnames}
\usepackage[colorlinks,
            linkcolor=black,
            urlcolor=black
            ]{hyperref}
\usepackage{fancyhdr}
\usepackage[top=1in, bottom=1in, left=1in, right=1in]{geometry}
\fancyhead{}
\fancyhead[LO,RE]{\leftmark}
\fancyhead[LE,RO]{泰国旅游基础攻略}
\fancyhead[CEO]{版权所有\ 禁止商用}
%\usepackage{titlesec}
\title{\textbf{泰国旅游基础攻略}}
\author{杨志鹏}
\begin{document}
\setcounter{page}{0}
\maketitle
\thispagestyle{empty}
\newpage
\pagestyle{fancy}
\tableofcontents
\paragraph{}
\noindent{}
\paragraph{}
\textbf{免责及版权声明}

本文不为任何行动提供担保责任。也不对任何用户使用阅读本文后实施的一切行动所造成的任何后果负责。

本文可非商业使用,但必须附上源地址,禁止移除作者名。商业使用,必须得到书面或其他许可。\\\\
\section{紧急电话与安全}
\begin{itemize}
\item 泰国报警电话:191(但警察英文水平很差)
\item 中国大使馆值班手机:0818452663(推荐)
\item 中国大使馆电话: 022457043 022450088
\item 中国大使馆24小时响应电话:0854833327(不推荐)
\end{itemize}

在曼谷的TUTU车上很容易被抢(即使是双肩包),有时候会出现人被拽下车子的情况。曼谷比较多飞车党,要注意安全,走路内侧较多人位置。
\section{可能需要准备的物品}
\begin{enumerate}
\item \textbf{2万泰铢或等值外币}(海关随机抽查)。
\item 雨伞等雨具(泰国北部地区的雨季时间是5月-10月)。
\item 相关城市地图(或电子地图)。
\item 单反(SD卡、备用SD卡)、卡片机、手机、ipad、移动电源、移动硬盘,各种充电器(泰国插头是2孔的,如果有3孔如笔记本就要用转换插头,很多旅馆只有一个插头,自带一个插座可以同时给多台设备供电);
\item 旅行证件:护照及签证(复印件)、电子机票(两份)、酒店预订单、相机购买发票凭证复印件,三张两寸照片备用、保险打印单;
\item 现金/卡类:现金(没带,直接ATM取)、信用卡、华夏银联卡、招行银联卡;
\item 洗漱用品:毛巾、沐浴露、洗发水/护发素、洗面奶、牙膏、牙刷 (泰国很多酒店不提供牙刷牙膏,最好自带)、剃须刀;
\item 皮肤护理:面霜、眼霜、面膜、防晒霜(太阳很毒,一定要SPF50+的)、梳子、镜子;
\item 生活用品:湿巾、纸巾、太阳眼镜、遮阳帽、雨伞、背包防水罩、相机防水袋(出海必备);
\item 衣物鞋帽:夏装3-4套、长衣长裤加外套(飞机上、机场里空调比较冷)、内衣裤、袜子、比基尼泳衣、人字拖、晾衣绳;
\item 药品清单:风油精、邦迪、消炎药(头孢)、感冒药、腹泻药、过敏体质建议带上开瑞坦(热带水果和食物易引发过敏)、晕船药(以防出海晕船)。
\end{enumerate}
\section{衣服}
泰国属于热带季风气候,常年温度不低于18℃。泰国全年可明显分为三季:
\begin{itemize}
\item 3月-5月气温最高,平均32℃~38℃,称为“热季”,空气干燥。
\item 6月至10月下旬,此为“雨季”,全年有85%的雨量集中在雨季,月平均温度维持在27、8℃左右。
\item 11月至次年2月是泰国一年之中最佳的季节——“凉季”,平均气温为19℃-26℃。
\end{itemize}
\section{其他注意事项}
\begin{enumerate}
\item \textbf{泰国没有单人落地签,没有泰国签证无法在中国海关出关。}
\item 华夏银行境外每天第一笔取款免手续费(华夏银行免手续费,但当地银行可能会收取手续费),机场可以取现。部分商店也可通过银联卡支付(手续费率视乎开卡行)。
\item 国内预先在银行兑换泰铢(可能需要预约)手续费更低。
\item 可以提前在淘宝购买泰国Happy手机卡,或在机场买AIS卡(上网速度较快,但往国内打电话较贵)。
\item 机场有免费泰国地图。
\item 泰国时间比北京时间晚一个小时。
\item 泰国可以使用谷歌地图(推荐)。
\item 小费:泰国的确是小费国家,但是并不意味着处处要给小费。比如,你在餐厅吃完饭,要先看看结账单里有没有服务费一项,如果收了服务费,你就不用给小费。另外在街边商贩那里吃饭,不需要给小费。快餐店也不需要给小费。小费最少给20铢,住酒店的时候小费可以放在床头。
\item 计程车很大可能会绕远路。从曼谷国际机场打的士时,跟机场管理人员报了地址后,他会写两联,一联给司机,一联给你。给你的那一联千万可以用来投诉,不可以给司机。从一个地方打的士到另一个地方,最好先问下酒店的服务员,他会给张地图标注好给你,最好也把价格问一下。
\item 在泰国离境的时候,有一对骗子试图装作丢钱向中国人行骗。男性骗子护照上的名字是魏希有。
\end{enumerate}
\section{相关应用}
\begin{enumerate}
\item \href{http://voyager2718.github.io/APK/com.bravolang.thai.apk}{学泰语 · 泰语会话入门 Android版}
\item \href{http://voyager2718.github.io/APK/off.guide.maps.thailand.apk}{泰国离线地图及指南 Android版}
\item \href{http://voyager2718.github.io/APK/travel.translator.apk}{旅行翻译官 Android版(需要下载离线句库和发音库)}
\item \href{http://qiugonglue.com/app/2}{曼谷攻略}
\item \href{http://qiugonglue.com/app/3}{布吉岛攻略}
\item \href{http://qiugonglue.com/app/4}{清迈攻略}
\end{enumerate}
\end{document}