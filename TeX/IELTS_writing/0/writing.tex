%!Mode::"TeX:UTF-8"
\documentclass[UTF8]{ctexart}
\usepackage[top=0.5in, bottom=1in, left=1in, right=1in]{geometry}
\usepackage{fancyhdr}
\title{IELTS writing 0}
\author{YANG Zhipeng}
\date{}
\begin{document}
\maketitle
\thispagestyle{empty}
\LARGE{\underline{Topic:}}

The leaders of organizations are often older people. But some people say that young people can also be leaders. What do you think?

\LARGE{\underline{Writing:}}

According to my observations, a huge proportion of leaders in most countries, such as China or Russia are older people.

I agree with these arrangements.

Some people hold an opinion that younger leaders may be smarter and have more enterprising spirits and would contribute more intelligent ideas and solutions.

I agree with this thinking.

However, depending on their ages, they may not have as much experience as older leaders. Moreover, experience holds an important position in our working life. Taking Chinese leaders as example, we can draw a conclusion that after experiencing many import positions in government, leaders will inevitably be old. And these experiences are important to one's career.

So I think that older leaders may be a better choice.
\end{document}
