%!Mode::"TeX:UTF-8"
\documentclass[UTF8]{ctexart}
\usepackage[top=0.5in, bottom=1in, left=1in, right=1in]{geometry}
\usepackage{fancyhdr}
%\renewcommand{\baselinestretch}{2}
\title{IELTS writings 0}
\date{}
\begin{document}
\pagestyle{plain}
\maketitle
\noindent
\LARGE{\underline{Topic:}}

\normalsize{In most countries multinational companies and their products are becoming more and more important.
This trend is seriously damaging our quality of life.}

\noindent
\LARGE{\underline{Writing:}}

\normalsize{The writer has tried to avoid repeating the same words too often in the answer.
Read the sample again and find synonyms or phrases later in the answer with similar meanings to the underlined words.

Multinational companies nowadays find it easy both to market their products all over the world and set up factories wherever they find it convenient. In my opinion this has had a harmful effect on our quality of life in three main areas.

The first area is their products. Supporters of globalization would argue that multinational companies make high-quality goods available to more people. While this may be true to some extent, it also means that we have less choice of products to buy. When powerful multinational companies invade local markets with their goods, they often force local companies with fewer resources to go out of business. In consequence, we are obliged to buy multinational products whether we like them or not.

This brings me to my second point. It is sometimes said that multinational companies and globalisation are making societies more open. This may be true. However, I would argue that as a result the human race is losing its cultural diversity. If we consumed different products, societies all over the world would be more varied. This can be seen by the fact that we all shop in similar multinational supermarkets and buy identical products wherever we live.

Thirdly, defenders of multinational companies often point out that they provide employment. Although this is undoubtedly true, it also means that we have become more dependent on them, which in turn makes us more vulnerable to their decisions. When, for example, a multinational decides to move its production facilities to another country, this has an adverse effect on its workers who lose their jobs.

All in all, I believe that if we as voters pressured our governments to make multinational companies more responsible and to protect local producers from outside competition, we could have the benefits of globalisation without its disadvantages.}

\noindent\rule[0.5ex]{\linewidth}{0.25pt}
\LARGE{\underline{Topic:}}

\normalsize{Successful sports professionals can earn a great deal more money than people in other important professions.
Some people think this is fully justified while others think it is unfair.

Discuss both these views and give your own opinion.}

\noindent
\LARGE{\underline{Writing:}}

\normalsize{As a result of constant media attention, sports professionals in my country have become stars and celebrities, and those at the top are paid huge salaries. Just like movie stars, they live extravagant lifestyles with huge houses and cars.

Many people find their rewards unfair, especially when comparing these super salaries with those of top surgeons or research scientists, or even leading politicians who have the responsibility of governing the country. However, sports salaries are not determined by considering the contribution to society a person makes, or the level of responsibility he or she holds. Instead, they reflect the public popularity of sport in general and the level of public support that successful stars can generate. So the notion of ‘fairness’ is not the issue.

Those who feel that sports stars’ salaries are justified might argue that the number of professionals with real talent are very few, and the money is a recognition of the skills and dedication a person needs to be successful. Competition is constant and a player is tested every time they perform in their relatively short career. The pressure from the media is intense and there is little privacy out of the spotlight. So all of these factors may justify the huge earnings.

Personally, I think that the amount of money such sports stars make is more justified than the huge earnings of movie stars, but at the same time, it indicates that our society places more value on sport than on more essential professions and achievements.}

\noindent\rule[0.5ex]{\linewidth}{0.25pt}
\LARGE{\underline{Topic:}}

\normalsize{In today's competitive world, many families find it necessary for both parents to go out to work. While some say the children in these families benefit from the additional income, others feel they lack support because of their parents' absence.}

\noindent
\LARGE{\underline{Writing:}}

\normalsize{In the past a typical family consisted of a father who went out to work and a mother who stayed at home and looked after the children. Nowadays, it is the norm for both parents to work. This situation can affect children both positively and negatively.

Some people think that the children of working parents are in an advantageous position where their parents are able to afford mere luxuries such as new clothes, video games or mobile phones. Proponents of this view argue that children are able to enjoy and experience more from life due to their parents' extra wealth, for example, by going on foreign holidays.

On the other hand, however, there are those who claim that when both parents work, their children do not get enough support and attention, meaning that the children might not do as well at school because there is no one at house to provide support with such things as homework or exam revision. The absence of a parent at home could make it easier for children to get involved in such things as drugs or undertake drinking.

When I was growing up, both my parents worked and I was always well provided for. On the other hand, I think that it would sometimes have been better if I could have seen more of my parents.

In conclusion, I believe that we cannot change the fact that both parents have to work nowadays. It is not an ideal situation, but if parents make time for their children in the evenings and at the weekends, then the children will not suffer in any way. It must be stated that the extra income generated by both parents working, makes for a much higher standard of living which benefits the whole family.}

\noindent\rule[0.5ex]{\linewidth}{0.25pt}
\LARGE{\underline{Topic:}}

\normalsize{If you could change one important thing about your hometown, what would you change?

Use reasons and specific examples to support your answer.}

\noindent
\LARGE{\underline{Writing:}}

\normalsize{If I could change one thing about my hometown, I think it would be the fact that there’s no sense of community here. People don’t feel connected, they don’t look out for each other, and they don’t get to know their neighbors.

People come and go a lot here. They change jobs frequently and move on. This means that they don’t put down roots in the community. They don’t join community organizations and they’re not willing to get involved in trying to improve the quality of life. If someone has a petition to put in a new street light, she has a very hard time getting a lot of people to sign. They don’t feel it has anything to do with them. They don’t get involved in improving the schools because they don’t think the quality of education is important to their lives. They don’t see the connection between themselves and the rest of their community.

People don’t try to support others around them. They don’t keep a friendly eye on their children, or check in on older folks if they don’t see them for a few days. They’re not aware when people around them may be going through a hard time. For example, they may not know if a neighbor loses a loved one. There’s not a lot of community support for individuals.

Neighbors don’t get to know each other. Again, this is because people come and go within a few years. So when neighbors go on vacation, no one is keeping an eye on their house. No one is making sure nothing suspicious is going on there, like lights in the middle of the night. When neighbors’ children are cutting across someone’s lawn on their bikes, there’s no friendly way of casually mentioning the problem. People immediately act as if it’s a major property disagreement.

My hometown is a nice place to live in many ways, but it would be much nice if we had that sense of community.}

\noindent\rule[0.5ex]{\linewidth}{0.25pt}
\LARGE{\underline{Topic:}}

\normalsize{It has been said, “Not every thing that is learned is contained in books.”

Compare and contrast knowledge gained from experience with knowledge gained from books. In your opinion, which source is more important? Why?}

\noindent
\LARGE{\underline{Writing:}}

\normalsize{“Experience is the best teacher” is an old cliche, but I agree with it. The most important, and sometimes the hardest, lessons we learn in life come from our participation in situations. You can’ learn everything from a book.

Of course, learning from books in a formal educational setting is also valuable. It’s in schools that we learn the information we need to function in our society. We learn how to speak and write and understand mathematical equations. This is all information that we need to live in our communities and earn a living.

Nevertheless, I think that the most important lessons can’t be taught; they have to be experienced. No one can teach us how to get along with others or how to have self-respect. As we grow from children into teenagers, no one can teach us how to deal with peer pressure. As we leave adolescence behind and enter adult life, no one can teach us how to fall in love and get married.

This shouldn’t stop us from looking for guidelines along the way. Teachers and parents are valuable sources of advice when we’re young. As we enter into new stages in our lives, the advice we receive from them is very helpful because they have already bad similar experiences. But experiencing our own triumphs and disasters is really the only way to learn how to deal with life.}

\noindent\rule[0.5ex]{\linewidth}{0.25pt}
\LARGE{\underline{Topic:}}

\normalsize{Nowadays, food has become easier to prepare. Has this change improved the way people live?

Use specific reasons and examples to support your answer.}

\noindent
\LARGE{\underline{Writing:}}

\normalsize{The twentieth century has brought with it many advances. With those advances, human lives have changed dramatically. In some ways life is worse, but mostly it is better. Changes in food preparation methods, for example, have improved our lives greatly.

The convenience of preparing food today is amazing. Even stoves have gotten too slow for us. Microwave cooking is much easier. We can press a few buttons and a meal is completely cooked in just a short time. People used to spend hours preparing an oven-cooked meal, and now they can use that time for other, better things. Plus, there are all kinds of portable, prepackaged foods we can buy. Heat them in the office microwave, and lunch at work is quick and easy.

Food preparation today allows for more variety. With refrigerators and freezers, we can preserve a lot of different foods in our homes. Since technology makes cooking so much faster, people are willing to make several dishes for even a small meal. Parents are more likely to let children be picky, now that they can easily heat them up some prepackaged macaroni and cheese on the side. Needless to say, adults living in the same house may have very different eating habits as well. If they don’t want to cook a lot of different dishes, it’s common now to eat out at restaurants several times a week.

Healthful eating is also easier than ever now. When people cook, they use new fat substitutes and cooking sprays to cut fat and calories. This reduces the risk of heart disease and high cholesterol. Additionally, we can buy fruits and vegetable fresh, frozen or canned. They are easy to prepare, so many of us eat more of those nutritious items daily. A hundred years ago, you couldn’t imagine the process of taking some frozen fruit and ice from the freezer, adding some low-fat yogurt from a plastic cup and some juice from a can in the refrigerator, and whipping up a low-fat smoothie in the blender!

Our lifestyle is fast, but people still like good food. What new food preparation technology has given us is more choices. Today, we can prepare food that is more convenient, healthier, and of greater variety than ever before in history.}
\end{document}
